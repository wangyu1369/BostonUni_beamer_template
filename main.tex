\documentclass[10pt,xcolor=table]{beamer}

% Use the preamble setting
\newtheorem{assumption}{Assumption}
% \newtheorem{theorem}{Theorem}
% \newtheorem{example}{Example}
% \newtheorem{remark}{Remark}
% \newtheorem{corollary}{Corollary}
 \newtheorem{proposition}{Proposition}
% \newtheorem{lemma}{Lemma}
% \newtheorem{condition}{Condition}
% \newtheorem{definition}{Definition}
\input{preamble.tex}
% !TEX root =  MCMC-VI-diagnostics-aistats.tex

\newcommand{\grad}{\nabla}

%\newcommand{\lt}{\mathopen{}\mathclose\bgroup\left}
%\newcommand{\rt}{\aftergroup\egroup\right}
%\newcommand{\m}{\middle}
%\newcommand{\lb}{\bigl}
%\newcommand{\lB}{\Bigl}
%\newcommand{\lbg}{\biggl}
%\newcommand{\lBg}{\Biggl}
%\newcommand{\rb}{\bigr}
%\newcommand{\rB}{\Bigr}
%\newcommand{\rbg}{\biggr}
%\newcommand{\rBg}{\Biggr}

%\DeclarePairedDelimiter\norm{\lVert}{\rVert}
%\DeclarePairedDelimiter\inner{\langle}{\rangle}
%\DeclarePairedDelimiter\ceil{\lceil}{\rceil}
%\DeclarePairedDelimiter\floor{\lfloor}{\rfloor}
%\DeclarePairedDelimiter\rbra{(}{)}
%\DeclarePairedDelimiter\cbra{\{}{\}}
%\DeclarePairedDelimiter\sbra{[}{]}

%--------------------------------------------------------------------------------------
% General
%--------------------------------------------------------------------------------------
\newcommand{\minop}{\mathbin{\wedge}}
\newcommand{\abscont}{\ll}  % absolutely continuous 
\newcommand{\nnreals}{\reals_{+}}
\newcommand{\cconj}[1]{\overline{#1}}  % complex conjugation 
\newcommand{\im}{\operatorname{im}}

\newcommand{\fracexp}[3]{\left(\frac{#1}{#2}\right)^{#3}}
\newcommand{\gammafrac}[2]{\frac{\Gamma(#1)}{\Gamma(#2)}}
\newcommand{\textder}[2]{\dee #1/\dee #2}

\newcommand{\LSI}[1]{\ensuremath{\mathrm{LS}(#1)}}
\newcommand{\lik}{f}
\newcommand{\loglik}{\mcL}
\newcommand{\mcL}{\mathcal{L}}

\newcommand{\onenorm}[1]{\norm{#1}_1} % L1 norm
\newcommand{\twonorm}[1]{\norm{#1}_2} % L2 norm

\newcommand{\benum}{\begin{enumerate}}
\newcommand{\eenum}{\end{enumerate}}
\newcommand{\kl}[2]{\mathrm{H}(#1 \mid #2)}
\newcommand{\TV}[2]{\textrm{TV}(#1, #2)}
\newcommand{\pwass}[4]{\mathrm{W}_{#1,#2}(#3, #4)}
\newcommand{\coupling}{\gamma}
%--------------------------------------------------------------------------------------
% Lp spaces
%--------------------------------------------------------------------------------------
\newcommand{\Lp}[1]{L^{#1}}
\newcommand{\Lpnorm}[2]{\norm{#2}_{\Lp{#1}}}
\newcommand{\Lpinner}[3]{\inner{#2}{#3}_{\Lp{#1}}}
\newcommand{\Lparg}[2]{L^{#1}(#2)}
\newcommand{\Lpnormarg}[3]{\staticnorm{#3}_{\Lparg{#1}{#2}}}
\newcommand{\Lpinnerarg}[4]{\inner{#3}{#4}_{\Lparg{#1}{#2}}}
\newcommand{\lpnorm}[2]{\norm{#2}_{#1}}
\newcommand{\lpinner}[3]{\inner{#2}{#3}_{#1}}



\newcommand{\metric}{m}
\newcommand{\mcmckernel}{p}
\newcommand{\Xtdist}[1]{\rho_{#1}}
\newcommand{\Yndist}[1]{\rho_{#1}}

%--------------------------------------------------------------------------------------
% Statistics
%--------------------------------------------------------------------------------------
\newcommand{\MAD}[2]{\operatorname{MAD}_{#1}({#2})}
%\newcommand{\MADveci}[2]{\operatorname{MAD}_{#1,#2}}
\newcommand{\mean}[2]{\mathrm{mean}_{#1}({#2})}
%\newcommand{\meanvec}[1]{\mu_{#1}}
%\newcommand{\meanveci}[2]{\mu_{#1,#2}}
%\renewcommand{\cov}[2]{\mathrm{cov}_{#1}({#2})}
\newcommand{\std}[2]{\mathrm{std}_{#1}({#2})}
\newcommand{\dev}[2]{\mathrm{dev}_{#1}({#2})}

\newcommand{\Var}{\mathrm{var}}
\def\eucnorm#1{|{#1}|} % A norm with 1 argument
\newcommand{\pwassSimple}[3]{\mathrm{W}_{#1}(#2, #3)}


\newcommand{\mainPE}{V}
\newcommand{\pspace}{E}  % parameter space

% Equality operators
%\mathtoolsset{centercolon}
\newcommand{\defined}{:=}
\newcommand{\xapprox}[1]{\underset{#1}{\approx}}
\newcommand{\medeq}{\!=\!}
\newcommand{\shorteq}{\!\!=\!\!}

%--------------------------------------------------------------------------------------
% SDEs
%--------------------------------------------------------------------------------------
\newcommand{\SDE}{\operatorname{SDE}}
\newcommand{\Wiener}[1]{W_{#1}}
\newcommand{\dWt}[1]{\dee \Wiener{#1}}
\newcommand{\Brownian}[2]{B_{#2}^{(#1)}}
\newcommand{\drift}{b}
\newcommand{\adrift}{\tilde\drift}
\newcommand{\potential}{U}
\newcommand{\apotential}{\tilde\potential}
\newcommand{\Ft}[1]{F_{#1}}
\newcommand{\dFt}[1]{\dee \Ft{#1}}
\newcommand{\allX}{\boldsymbol{X}}
\newcommand{\Xt}[1]{X_{#1}}
\newcommand{\dXt}[1]{\dee \Xt{#1}}
\newcommand{\allY}{\boldsymbol{Y}}
\newcommand{\Yn}[1]{Y_{#1}}
\newcommand{\dt}{\dee t}
\newcommand{\df}{\dee\mkern-1.5mu f}
\newcommand{\domega}{\dee \omega}
\newcommand{\transop}[1]{P_{#1}} % transition semigroup
\newcommand{\gen}{\mcL} % generator
\newcommand{\adgen}{\gen^{*}} % adjoint of generator
\newcommand{\gent}[1]{\gen_{#1}} % time-dependent generator
\newcommand{\adgent}[1]{\adgen_{#1}} % time-dependent generator
\newcommand{\Dform}{\mcE} % Dirichlet form
\newcommand{\cdc}{\Gamma} % carré du champ 
\newcommand{\icdc}{\Gamma_{2}} % iterated carré du champ 
\newcommand{\probspace}{\mcP}
\newcommand{\pprobspace}[1]{\mcP_{#1}}



%--------------------------------------------------------------------------------------
% Bayesian stuff
%--------------------------------------------------------------------------------------
\newcommand{\xvec}{x}
\newcommand{\xveci}[1]{x_{#1}}
\newcommand{\Xvec}{X}
\newcommand{\Xveci}[1]{X_{#1}}
\newcommand{\dx}{\dee\xvec}
%\newcommand{\targetdist}{\pi}
\newcommand{\basemeas}{\lambda}
\newcommand{\vardist}{\xi}
\newcommand{\varfamily}{\mcQ}
\newcommand{\approxdist}{\hat\targetdist}
\newcommand{\estdist}{\hat\targetdist}
\newcommand{\targetdistunnormed}{\targetdist'}
\newcommand{\mainmeas}{\nu}
\newcommand{\estmeas}{\eta}
\newcommand{\discrepancy}[2]{{\mathrm{D}_{#1}(#2)}}
\newcommand{\discrepancyAdjusted}[2]{{\mathrm{\overline D}_{#1}(#2)}}


\usepackage{mathtools}
\usepackage{thmtools}
%%% MLE estimation with sample size N
\newcommand{\MLEN}{\widehat{\theta}^{(N)}}
\newcommand{\MLE}{\widehat{\theta}}
\newcommand{\Norm}{\mathcal{N}}
\newcommand{\dist}{\sim}
\newcommand{\distas}{\sim}
\newcommand{\distiid}{\overset{\text{iid}}{\distas}}

\newcommand{\obsSpace}{\mathbb{X}}
\newcommand{\loss}{\mathcal{L}}
\newcommand{\hess}{\widehat{H}}
\newcommand{\reg}{\mathcal{R}}

\DeclareMathOperator{\Tr}{Tr}
\DeclarePairedDelimiter\norm{\lVert}{\rVert}
\DeclarePairedDelimiter\inner{\langle}{\rangle}
\DeclarePairedDelimiter\ceil{\lceil}{\rceil}
\DeclarePairedDelimiter\floor{\lfloor}{\rfloor}
\DeclarePairedDelimiter\rbra{(}{)}
\DeclarePairedDelimiter\cbra{\{}{\}}


%%%%%%%%%%%%%%%%%%%%%%%%%%%%%%%%%%
%%% Probability and Statistics %%%
%%%%%%%%%%%%%%%%%%%%%%%%%%%%%%%%%%
% text shortcuts
\newcommand{\iid}{\textrm{i.i.d.}\@\xspace}
\newcommand{\as}{\textrm{a.s.}\@\xspace}
\newcommand{\aev}{\textrm{a.e.}\@\xspace}

% convergence
\newcommand{\convas}{\overset{a.s.}{\to}}
\newcommand{\convp}{\overset{p}{\to}}
\newcommand{\convd}{\overset{d}{\to}}
\newcommand{\eqd}{\overset{d}{=}}
\newcommand{\eqas}{\overset{a.s.}{=}}

%\newcommand{\distiid}{\overset{\text{iid}}{\dist}}

\newcommand{\distNamed}[1]{{\sf{#1}}}
\newcommand{\distNorm}{\mathcal{N}}
\newcommand{\distBern}{\distNamed{Bern}}
\newcommand{\distInvGam}{\distNamed{InvGam}}
\newcommand{\distind}{\overset{\textrm{\tiny\textrm{indep}}}{\distas}}

%%% Put project-specific definitions here %%%
%%%%%%%%%%%%%%%%%%%%%%%%%%%%%%%%%%%%%%%%%%%%%%
%%% Common Sets and Set Operations
%%%%%%%%%%%%%%%%%%%%%%%%%%%%%%%%%%%%%%%%%%%%%%
\newcommand{\reals}{\ensuremath{\mathbb{R}}}
\newcommand{\xreals}{\overline{\reals}}
\newcommand{\ints}{\ensuremath{\mathbb{Z}}}
\newcommand{\rats}{\ensuremath{\mathbb{Q}}}
\newcommand{\nats}{\ensuremath{\mathbb{N}}}
\newcommand{\comps}{\ensuremath{\mathbb{C}}}
\newcommand{\msrs}{\mathcal{M}}
\newcommand{\pmsrs}{\mathcal{M}_1}
\newcommand{\union}{\bigcup}
\newcommand{\inter}{\bigcap}
\newcommand{\vol}{\operatorname{vol}}
\newcommand{\diam}{\operatorname{diam}}
\newcommand{\cl}{\operatorname{cl}}
\newcommand{\spann}{\operatorname{span}}
\newcommand{\bdry}{\partial}
\newcommand{\cone}{\operatorname{cone}}
\newcommand{\conv}{\operatorname{conv}}

% unary/functions
\renewcommand{\Pr}{\mathbb{P}}  % probability
\newcommand{\E}{\mathbb{E}}	% expectation
\newcommand{\EE}{\mathbb{E}}	% expectation
\newcommand{\Law}{\mcL} 
\newcommand{\var}{\operatorname{Var}}	% variance
\newcommand{\cov}{\operatorname{Cov}}	% covariance
\newcommand{\corr}{\operatorname{Corr}}	% correlation
\newcommand{\supp}{\operatorname{supp}} %support

%Proof clever ref names

%\newtheorem{assumption}{Assumption}
%\newtheorem{theorem}{Theorem}
%\newtheorem{example}{Example}
%\newtheorem{remark}{Remark}
%\newtheorem{corollary}{Corollary}
%\newtheorem{proposition}{Proposition}
%\newtheorem{lemma}{Lemma}
%\newtheorem{condition}{Condition}
%\newtheorem{definition}{Definition}

%%%%%%%%%%%%%%%%%%%%%%%
%%% General Purpose %%%
%%%%%%%%%%%%%%%%%%%%%%%
\newcommand{\tuple}[1][]{\left\langle #1 \right\rangle}
\newcommand{\emptystr}{\text{\o}}
\newcommand{\ind}{\mathds{1}} % indicator function
\newcommand{\sgn}{\operatorname{sgn}} % sign function
\newcommand{\theset}[1]{\lbrace #1 \rbrace}
\newcommand{\sn}[2]{\ensuremath{#1\!\times\!10^{#2}}}	% #1 x 10^{#2}

\newcommand{\st}{\,:\,}
\newcommand{\given}{\mid}


%%%%%%%%%%%%%%%%%
%%% Shortcuts %%%
%%%%%%%%%%%%%%%%%
\newcommand{\eps}{\epsilon}
\newcommand{\veps}{\varepsilon}


% Equality operators
\newcommand{\defas}{:=}
\newcommand{\defines}{=:}




\def\argmax{\operatornamewithlimits{arg\,max}}
\def\argmin{\operatornamewithlimits{arg\,min}}
\def\esssup{\operatornamewithlimits{ess\,sup}}
\def\essinf{\operatornamewithlimits{ess\,inf}}

%\newcommand{\reals}{\ensuremath{\mathbb{R}}}
\newcommand{\dee}{\mathrm{d}}
%\newcommand{\distas}{\sim}
\newcommand{\mcF}{\mathcal{F}}

\newcommand{\wheretext}[1]{\textcolor{magenta}{[#1]}}

\newcommand{\diagnosticName}{TADDAA\xspace}

\newcommand{\posterior}{\Pi}
\newcommand{\posteriorDensity}{\pi}

\newcommand{\initialDensity}{\hat{\pi}^{(0)}}
\newcommand{\improvedDensityAt}[1]{\hat{\pi}^{(#1)}}
\newcommand{\improvedDensity}{\improvedDensityAt{T}}

%\newcommand{\proposal}[3]{Q_{#3}(#1, \ifstrempty{#2}{\cdot}{\dee{#2}})}  % MCMC proposal kernel density (old value, new value, parameter)
%\newcommand{\mhkernel}[3]{P_{#3}(#1, \ifstrempty{#2}{\cdot}{\dee{#2}})}  % MCMC kernel (new value, old value, parameter)

\newcommand{\proposal}[3]{Q_{#3}(#1, {\dee{#2}})}  % MCMC proposal kernel density (old value, new value, parameter)
\newcommand{\mhkernel}[3]{P_{#3}(#1, {\dee{#2}})}  % MCMC kernel (new value, old value, parameter)


\newcommand{\proposalDensity}[3]{q_{#3}(#1, #2)}  % MCMC proposal kernel density (old value, new value, parameter)
\newcommand{\mhkernelDensity}[3]{p_{#3}(#1, #2)}  % MCMC kernel (new value, old value, parameter)
\newcommand{\proposalParam}{\psi}
\newcommand{\acceptprob}[2]{\alpha(#1, #2)}
\newcommand{\rejectprob}[2]{r_{#2}(#1)}

\newcommand{\state}{x}
\newcommand{\proposedState}{y}
\newcommand{\stateDim}{d}
\newcommand{\stateDimInd}{i}

\newcommand{\mcState}{X}
\newcommand{\mcProposedState}{Y}
\newcommand{\mcIters}{T}
\newcommand{\mcIterInd}{t}

%sequence of samples
\newcommand{\CurrentSamples}{X_{1:N}^{(t)}}
\newcommand{\CurrentProposals}{Y_{1:N}^{(t)}}

%single sample and proposal
\newcommand{\CurrentSingleSample}{X_{j}^{(t)}}
\newcommand{\CurrentSingleSamplePre}{X_{j}^{(t-1)}}
\newcommand{\CurrentSingleProposal}{Y_{j}^{(t)}}
\newcommand{\CurrentSingleProposalPre}{Y_{j}^{(t-1)}}
\newcommand{\NewSingleSample}{X_{j}^{(t+1)}}
\newcommand{\NewSingleProposal}{Y_{j}^{(t+1)}}

%step size
\newcommand{\CurrentStepsize}{h_{N}^{(t)}}
\newcommand{\PreviousStepsize}{h_{N}^{(t-1)}}
\newcommand{\NewStepsize}{h_{N}^{(t+1)}}
\newcommand{\LimitingStepsizePre}{\bar{h}^{(t-1)}}
\newcommand{\LimitingStepsize}{\bar{h}^{(t)}}

%empirical measure
\newcommand{\EmpiricalMeasureX}{\nu_{N}^{(t)}}
\newcommand{\LimitingMeasureXt}{\bar{\nu}^{(t)}}


\newcommand{\EmpiricalMeasureXY}{\xi_{N}^{(t)}}
\newcommand{\EmpiricalMeasureXYPre}{\xi_{N}^{(t-1)}}
\newcommand{\LimitingMeasureXY}{\bar{\xi}^{(t)}}
\newcommand{\LimitingMeasureXYPre}{\bar{\xi}^{(t-1)}}

\newcommand{\EmpiricalMeasureXYTilde}{\tilde{\xi}_N^{(t)}}

% filtration
\newcommand{\Filtrationcurrent}{\mathcal{F}_{N}^{(t)}}
\newcommand{\FiltrationcurrentPre}{\mathcal{F}_{N}^{(t-1)}}


% Title and logo
\title{Enhancing the Reliability of General-Purpose Algorithms for Approximate Bayesian Inference}
%\subtitle{Prospectus Defense}
%\date{\today}
\author{Yu Wang}

% Add bibliography file
\bibliography{reference}

\begin{document}
% Make the title
{
% Uncomment the following to dismiss the transparent background crest
\setbeamertemplate{background}{\tikz[remember picture,overlay]\node[opacity=0.04] at (current page.center) {\includegraphics[height=0.8\paperheight,keepaspectratio]{BostonUni_Crest.eps}};}
\maketitle
}

% Setting all titles to be centered
\setbeamertemplate{frametitle}[default][center]

% Table of content
%\begin{frame}[plain]{Table of Contents}
%  \setbeamertemplate{section in toc}[sections numbered]
%  \tableofcontents[hideallsubsections]
%\end{frame}

%\begin{frame}{Bayesian Approximation}
%\begin{equation*}
%	\pi(\theta \mid Y)=\frac{p(Y \mid \theta) \pi_0(\theta)}{Z}
%\end{equation*}
%	\begin{itemize}
%		\item We want to learn about $\pi$, typically by calculating expectations
%		$$\EE\{f(\Xt{})\} \defined \pi(f) \defined \int f(\xvec) \pi(\dx).$$ \pause
%		\item However, in general, the expectations can’t be done exactly. \pause
%		\item Approximate inference: Markov chain Monte Carlo (MCMC) constructs a Markov chain $\Yn{0}, \Yn{1}, \Yn{2}, \dots,$ such that $\Yndist{n}$
%		\[
%		\sum_{n=1}^{N} f(\Yn{n}) / N \convas \mainmeas(f) \quad\text{for } N \to \infty. \label{eq:SLLN}
%		\]
%	\end{itemize}
%\end{frame}

\begin{frame}{Bayesian Approximation}
	\begin{equation*}
		\pi\left(\theta \mid \{Y_{i}\}_{i=1}^{N} \right)=\frac{\prod_{i=1}^{N} p(Y_{i} \mid \theta) \pi_0(\theta)}{Z}
	\end{equation*}
	\begin{itemize}
		\item We want to learn about $\pi$, typically by calculating expectations
		$$\EE\{f(\Xt{})\} \defined \pi(f) \defined \int f(\xvec) \pi(\dx).$$ \pause
		\item However, in general, the expectations can’t be done exactly. \pause
		\item Approximate inference: Markov chain Monte Carlo (MCMC) constructs a Markov chain $\Yn{0}, \Yn{1}, \Yn{2}, \dots,$ such that $\Yndist{n} \to \pi$
		\[
		\sum_{n=1}^{N} f(\Yn{n}) / N \convas \pi(f) \quad\text{for } N \to \infty. \label{eq:SLLN}
		\]
	\end{itemize}
\end{frame}

\begin{frame}{Challenges in Modern Approximate Bayesian Inference}
	\begin{itemize}
		\item \textbf{Challenges:}
		\begin{itemize}
			\item high-dimensional $\theta \in \reals^{D}$, $D$ is large.
			\item complex relationship $p(Y_{i}|\theta)$.
			\item large-sclae dataset $\{Y_{i}\}_{i=1}^{N}$. 
		\end{itemize}\pause
		\item \textbf{Problems of MCMC:}
		\begin{itemize}
			\item Slow convergence and mixing in high-dimensional spaces.
			\item Computational cost of likelihood evaluation is proportional to the dataset size.
			\item Diagnosing convergence is harder in high dimensions.
		\end{itemize} \pause
		\item \textbf{Mitigations:}
		\begin{itemize}
			\item Variational Inference (VI).
			\item Subsampling methods (e.g. Stochastic Gradient Langevin Dynamics (SGLD)).
		\end{itemize}
	\end{itemize}
\end{frame}

%\begin{frame}{Overview}
%	\begin{itemize}
%		\item A priori finite-time, finite-data guarantees:\\
%		A Unifying Framework for Understanding General-purpose Bayesian Posterior Approximation Methods, \\
%		Huggins, Kasprzak, \textbf{Wang}, Campbell, Broderick. In prep
%		\item A post hoc quality check for VI: \\A Targeted Accuracy Diagnostic for Variational Approximations (TADDAA), \textbf{Wang}, Kasprzak, Huggins. AISTATS 2023.
%		\item Uncertainty quantification for Subsampling methods:\\
%		Stationary Analysis of Fixed Learning Rate Stochastic Gradient Algorithms, \textbf{Wang} $\&$ Huggins. (Under Review)
%	\end{itemize}
%\end{frame}

\begin{frame}{Overview}
	\begin{itemize}
		\item \textcolor{gray}{A priori finite-time, finite-data guarantees:\\
		A Unifying Framework for Understanding General-purpose Bayesian Posterior Approximation Methods, \\
		Huggins, Kasprzak, \textbf{Wang}, Campbell, Broderick. In prep} \pause
		\item \textbf{A post hoc quality check for VI}: \\A Targeted Accuracy Diagnostic for Variational Approximations (TADDAA), \textbf{Wang}, Kasprzak, Huggins. AISTATS 2023.
		\item \textbf{Uncertainty quantification for Subsampling methods}:\\
		Stationary Analysis of Fixed Learning Rate Stochastic Gradient Algorithms, \textbf{Wang} $\&$ Huggins. (Under Review)
	\end{itemize}
\end{frame}

% Section 1
\section{TADDAA}

\begin{frame}
  \frametitle{Markov chain Monte Carlo (MCMC) }
  MCMC sampling methods provide a general-purpose framework for obtaining samples that are asymptotically exact.
  \begin{itemize}
  \item \textbf{Proposal distribution:} $\proposal{\state}{\proposedState}{\proposalParam}$ parameterized by $\proposalParam$ with current state $\state$ and corresponding density $\proposalDensity{\state}{\proposedState}{\proposalParam}$.
  \item \textbf{Metropolis--Hastings (MH) correction:} to construct a Markov kernel with the desired stationary distribution $\pi$, a proposed state 
$\mcProposedState \distas Q_{\psi}(x, \cdot)$ 
is accepted with probability 
\[ \label{eq:MH-accept}
\acceptprob{\state}{\mcProposedState} 
= \min \left\{1, \frac{\posteriorDensity(\mcProposedState) \proposalDensity{\mcProposedState}{\state}{\proposalParam}}{\posteriorDensity(\state) \proposalDensity{\state}{\mcProposedState}{\proposalParam}}\right\}. 
\]
  \end{itemize}
\end{frame}

\begin{frame}{Variational Inference (VI)}
Variational inference (VI) provides a potentially faster alternative to MCMC when models are complex and/or the dataset size is large.
\[
\hat{\posteriorDensity}   = \argmin_{\xi \in \mathcal{Q}} \mathcal{D}_{\posteriorDensity}(\xi).
\]
\begin{itemize}
%     \item VI aims to minimize some measure of discrepancy $\mathcal{D}_{\mainmeas}(\cdot)$ in a tractable family $\mathcal{Q}$ of potential
% approximating distributions
% \[ \hat{\mainmeas}   = \argmin_{\xi \in \mathcal{Q}} \mathcal{D}_{\mainmeas}(\xi). \]
    \item Variational family $\mathcal{Q}$: we are able to efficiently calculate
expectations of interest (e.g. mean and variance).  
    \item Measure of discrepancy $\mathcal{D}_{\mainmeas}(\cdot)$: the canonical choice is \emph{Kullback--Leibler (KL) divergence} out of convenience. 
\[
\mathcal{D}_{\posteriorDensity}(\xi) = \mathrm{KL}(\xi \mid \posteriorDensity) :=\int \log \left(\frac{\mathrm{d} \xi}{\mathrm{d} \posteriorDensity}\right) \mathrm{d} \xi.
\]
\end{itemize}
\end{frame}

\begin{frame}{Related Works}
\begin{itemize}
    \item Existing evaluation tools:
    \begin{itemize}
        \item Evidence Lower Bound (ELBO).
        \item Kernel Stein Discrepancy (KSD).
        \item Pareto smoothed importance sampling (PSIS) $\hat{k}$.
    \end{itemize}
    \item Problems:
    \begin{itemize}
        \item Lack interpretability.
        \item Not applicable in high-dimensional parameter spaces.
        \item Don't support marginal checks.
    \end{itemize}
\end{itemize}
\end{frame}

\begin{frame}{TADDAA:Intuition}
	
  We want to quantify approximation error $\varepsilon^{(0)}(\mcF) \defas \mcF(\initialDensity) - \textcolor{red}{\mcF(\posteriorDensity)}$ for a posterior functional of interest $\mcF$ such as a mean or variance.
  \begin{columns}
    \begin{column}{.44\textwidth}
      \begin{figure}[t]
        \includegraphics[width=\textwidth]{intuition_plot.pdf}
        \caption{$\hat{\pi}^{(T)}$ significantly different from $\hat{\pi}^{(0)}$ $\Rightarrow$ $\hat{\pi}^{(0)}$ far from $\pi$.}
      \end{figure}
    \end{column}
    \begin{column}{.6\textwidth}
        \begin{itemize}
        \item For another approximation $\hat{\pi}^{(T)}$ closer to taget posteriror $\pi$, $\varepsilon^{(T)}(\mcF) \defas \mcF(\improvedDensity) - \textcolor{red}{\mcF(\posteriorDensity)}$
        \item 
        \begin{equation*}
		\begin{aligned}
			&\varepsilon^{(0)}(\mcF)
			\ge |\mcF(\initialDensity) - \mcF(\improvedDensity)| \\
			&\ge \begin{cases}
				0 & \text{if $\ell_{\mcF} \le 0 \le u_{\mcF}$} \\
				\min(|\ell_{\mcF}|, |u_{\mcF}|) & \text{otherwise} 
			\end{cases} \\
			&= \mathbf{1} \left\{0 \notin (\ell_{\mcF},  u_{\mcF})\right\} \times \min(|\ell_{\mcF}|, |u_{\mcF}|) \\
			&\defines  B_{\mcF}. 
		\end{aligned}
		\end{equation*}
        \end{itemize}
    \end{column}
  \end{columns}
  
\end{frame}


\begin{frame}{TADDAA:Input}
\begin{columns}
  \begin{column}{0.5\linewidth}
\begin{figure}
  \begin{center}
    \includegraphics[width = \textwidth]{vi_samples.pdf}
  \end{center}
\end{figure}
  \end{column}
  \begin{column}{0.5 \linewidth}

       %\textbf{\textcolor{blue}{1.Input:}}
       % \heading{\textcolor{blue}{Input:}}
        \begin{itemize}
            \item log density of the target $\pi$
            \item approximating distribution $\hat{\pi}^{(0)}$
            \item functional of interest $\mathcal{F}$ (e.g. marginal mean)
            \item transition kernel $K_h(x, \mathrm{~d} y)$ (e.g. Barker, HMC) %and $\psi^{(0)}$
            \item number of Markov chains $N$ and iterations $T$
        \end{itemize}
  \end{column}
\end{columns}
\end{frame}

\begin{frame}{TADDAA:Run MCMC with inter-chain adaptation (INCA)}
  \begin{columns}
  \begin{column}{0.5\linewidth}
\begin{figure}
  \begin{center}
    \includegraphics[width = \textwidth]{improved_samples.pdf}
  \end{center}
\end{figure}
  \end{column}
  \begin{column}{0.6 \linewidth}
        %\textbf{\textcolor{blue}{2.Run MCMC with inter-chain adaptation:}}
        % \heading{\textcolor{blue}{2.Run MCMC with inter-chain adaptation (INCA):}} %use for loop
        % \begin{itemize}
        %     \item $X_{j}^{(t+1)} \sim K_{\psi^{(t)}}\left(X_{j}^{(t)}, \cdot\right)$
        %     \item $\psi^{(t+1)}=H\left(\psi^{(1: t)}, X_{1: N}^{(1: t+1)}, Y_{1: N}^{(1: t+1)}\right)$
        % \end{itemize}
        \textbf{\quad for} $t=0$ to $T-1$ \textbf{do}:\\
        \textbf{\quad\qquad for} $j=1$ to $N$ \textbf{do}:
        \text{\qquad\qquad $X_j^{(t+1)} \sim K_{h^{(t)}}(X_j^{(t)}, \cdot)$}\\
            % & \qquad\qquad \alpha_j^{(t)}=\alpha(X_j^{(t)}, Y_j^{(t)}) \\
            % & \qquad\qquad X_j^{(t+1)}= \begin{cases}Y_j^{(t)}, & \text { with probability } \alpha_j^{(t)} \\
            % X_j^{(t)}, & \text { with probability } 1-\alpha_j^{(t)}\end{cases}
            % & \qquad\qquad \text{accept} $ Y_j^{(t+1)}$ with probability $\alpha_j^{(t)}$
        \textbf{\quad\qquad end for}\\
        \text{\quad\qquad update step-size $h^{(t+1)}$ using INCA.}\\
        \text{\quad\textbf{end for}}
  \end{column}
  \end{columns}
\end{frame}

\begin{frame}{TADDAA:Compute error lower bounds and reliability check}
\begin{columns}[T]
  \begin{column}{0.5\linewidth}
\begin{figure}[p]
    \centering 
    \includegraphics[width = \textwidth]{run_tests.pdf}
\end{figure}
  \end{column}
  \begin{column}{0.5 \linewidth}
        %\textbf{\textcolor{blue}{3.Compute error lower bounds and reliability check:}}
        % \heading{\textcolor{blue}{3.Compute error lower bounds and reliability check:}}
        \begin{itemize}
                \item Compute correlation check $\rho_{\max }^2(T)$
                \item Compute a confidence interval $(\ell_{\mathcal{F}}, u_{\mathcal{F}})$ for $\mathcal{F}(\hat{\pi}^{(0)})-\mathcal{F}(\hat{\pi}^{(T)})$ based on $X_{1: N}^{(0)}$ and $X_{1: N}^{(T)}$
                \item Compute lower bound $B_{\mathcal{F}}$
        \end{itemize}
  \end{column}
  \end{columns}
\end{frame}

% \begin{frame}{Functional of interest $\mathcal{F}$}
% % For $t \in \{0,\dots, T\}$, define the mean vector
% % $\mu^{(t)} = (\mu_{1}^{(t)}, \ldots, \mu_{d}^{(t)})^{\top} \defas  \EE_{X \sim \improvedDensityAt{t}}(X)$, 
% % %$\Sigma^{(t)} \defas \cov_{X \sim \improvedDensityAt{t}} (X)$,
% % and marginal standard deviations $\sigma^{(t)}_{i} \defas \operatorname{stdev}_{X \sim \improvedDensityAt{t}}(X_{i})$ for $i = 1,\dots,d$.
% \begin{itemize}
%     \item Marginal mean $\mcF_{\text{mean}}^{i}(\posteriorDensity) \defas \mu_{i}(\posteriorDensity) \defas \int x_{i}\,\posteriorDensity(\dee x)$
% \end{itemize}
% \end{frame}

\begin{frame}{Transition kernel $K_h(x, \mathrm{~d} y)$}
Let $x \in \reals^{d}$ denote the current state, 
$h \in \reals_{+}$ the step size, and $G \in \reals^{d \times d}$ a positive semi-definition preconditioning matrix. 
\begin{itemize}
    \item Random Walk Metropolis-Hasting (RWMH). 
    \item Metropolis-adjusted Langevin algorithm (MALA).
    \item Hamiltonian Monte Carlo (HMC).
    \item \textbf{Barker Proposal (recommended choice)}: robust to precise step size and acceptance rate. 
\end{itemize}
\end{frame}

\begin{frame}{Step size $h$}
All four kernels rely on a step-size parameter $h^{(t)}$, to guarantee superior sampling efficiency when dimension $d$ is large, $h^{(t)}$ should be adapted according to \emph{inter-chain adaptation} (INCA) and \emph{optimal scaling}.
    \begin{itemize}
        \item Step size adaption: 
            \begin{itemize}
                \item Generate proposals 
$Y_{j}^{(t+1)} \distas Q_{\proposalParam^{(t)}}(X_{j}^{(t)}, \cdot)$, then accept with probability $\alpha^{(t)}_{j}$. 
                \item \[
\label{eq:adaptive step size}
\psi^{(t+1)}= \psi^{(t)} +  \frac{1}{\sqrt{t+1}}(\bar\alpha^{(t)} - \bar{\alpha}_{*}),
\]
where $\psi^{(t)} =\log h^{(t)}$ and $\bar{\alpha}_{*}$ is optimal asymptotic acceptance. 
            \end{itemize} \pause
        \item Optimal initial step size $h^{(0)}$ and $\bar{\alpha}_{*}$:
        \begin{itemize}
            \item RWMH: $h^{(0)} = 2.4^2/d$, $\bar{\alpha}_{*} = 0.234$. 
            \item MALA and Barker: $h^{(0)} = {2.4^2}/{d^{1/3}}$, $\bar{\alpha}_{*} = 0.576$.
            \item HMC: $h^{(0)} = {2.4^2}/{d^{1/4}}$, $\bar{\alpha}_{*} = 0.4$.
        \end{itemize} \pause 
        \item \textcolor{red}{Problem:} INCA introduces dependence between the Markov chains, which could invalidate the statistical tests and confidence intervals. 
    \end{itemize}
\end{frame}

\begin{frame}{Number of Markov chains $N$}
% The number of iterations has to be large enough that the improved distribution $\improvedDensity$ is sufficiently different from a poor approximating distribution $\initialDensity$.
The number of Markov chains must be sufficiently large that the confidence intervals are small enough to detect meaningful errors.
Hence, if the user's tolerance is $\delta_{\text{mean}}$ for relative mean error and $\delta_{\text{var}}$ for log variance error, then 
\[
\begin{aligned}
	\label{eq:sample-size}
	\textstyle  N &= \textstyle  \max \left( N_{\text{mean}}, N_{\text{variance}}\right),
\end{aligned}
\]
where 
\[
\begin{aligned}
\textstyle N_{\text{mean}} &\defas \textstyle  \min \left\{ n \in \nats : \frac{t_{n-1}(\alpha / 2)}{\sqrt{n}} \leq \delta_{\text{mean}} \right\},  \\
\textstyle  N_{\text{variance}} &\defas  \textstyle  \min \left\{ n \in \nats :  \log\left( \frac{\chi^{2}_{n-1}(1-\alpha/2)}{\chi^{2}_{n-1}(\alpha/2)}\right) \leq \delta_{\text{var}} \right\}.
\end{aligned}
\]
\end{frame}

\begin{frame}{Number of iterations $T$}
Markov chain requires $\Theta(d^{\gamma})$ iterations to mix according to theory of optimal scaling. 
\begin{itemize}
    \item For RWMH, MALA, Barker: $T = \lfloor c d^{1/3} \rfloor$.
    \item For HMC: $T = \lfloor c d^{1/4}/L \rfloor, $
    where $L$ is the number of leapfrog steps in HMC. 
\end{itemize} \pause
\textbf{Remark}
\begin{itemize}
    \item Based on our ablation studies, $c= 50$ is a reasonable choice. 
    \item Theories \footnote{Bhatia, Kush, et al. Statistical and computational trade-offs in variational inference: A case study in inferential model selection.} suggest computational cost of TADDAA is comparable to VI:
    \begin{itemize}
        \item Computational cost for VI: $\Theta(d^{1/3})$.
        \item Computational cost for MALA and Barker: $\Theta(d^{1/3})$.
        \item Computational cost for HMC: $\Theta(d^{1/4})$.
    \end{itemize}
\end{itemize}
\end{frame}

\begin{frame}{A Reliability Check for the Diagnostic}
The reliability of TADDAA depends on the mixing behavior of the Markov chains: 
\begin{itemize}
    \item If the Markov chains are mixing well, we expect the diagnostic results can be trusted.
    \item If the Markov chains are not mixing well, diagnosis of a poor approximation can still be trusted but diagnosis of a good approximation may not be reliable. \footnote{In the latter case, we should consider increasing the length of Markov chains or otherwise improving the Markov kernel.}
\end{itemize}
\end{frame}

\begin{frame}{A Reliability Check for the Diagnostic (Continued)}
% Instead, we propose to leverage the fact that we have many nearly independent chains to compute the correlation between the initial and final values of each chain.
% If the chains are mixing, then this correlation will be close to zero. 
% Specifically we propose to use the worst-case correlation coefficient

\begin{itemize} 
    \item We propose to use the worst-case correlation coefficient
$
\rho^{2}_{\max}(T)
\defas \max_{i} \rho^{2}_{i}(T),
%&= \max_{k}\frac{\sum_{j=1}^{N} ( X_{j,k}^{(0)}-\hat{\mu}_{k}^{(0)})( X_{j,k}^{(T)}-\hat{\mu}_{k}^{(T)})}{\sum_{j=1}^{N}( x_{j,k}^{(0)}-\hat{\mu}_{k}^{(0)})^{2} ( x_{j,k}^{(T)}-\hat{\mu}_{k}^{(T)})^{2}},
$
where 
\[
\begin{aligned}
	% \rho^{2}\Big(X_{1:N,k}^{(0)},  X_{1:N,k}^{(T)}\Big)
	\textstyle \rho^{2}_{i}(T)
	\defas  \frac{\sum_{j=1}^{N} ( X_{j,i}^{(0)}-\hat{\mu}_{i}^{(0)})( X_{j,i}^{(T)}-\hat{\mu}_{i}^{(T)})}{\sqrt{\sum_{j=1}^{N}( X_{j,i}^{(0)}-\hat{\mu}_{i}^{(0)})^{2}} \sqrt{(\sum_{j=1}^{N} X_{j,i}^{(T)}-\hat{\mu}_{i}^{(T)})^{2}}}.
\end{aligned}
\]
    \item Check passes: $\rho^{2}_{\max}(T) < 0.1$.
    % \item $\rho^{2}_{\max}(T)$ large, then the check passes.
\end{itemize}
\end{frame}

\begin{frame}{Asymptotic Independence of Adapted Markov Chains}
The Markov chains $\CurrentSamples$ are not independent once $t > 1$, 
so the final samples $X_{1:N}^{(T)}$ violate the independence requirement of statistical tests. \pause
\begin{definition}
	Let $X_{N,1:N} = (X_{N,1}, \dots, X_{N,N})$ denote a random vector.
	The sequence of random vectors $\{ X_{N,1:N} \}_{N=1}^{\infty}$ is \emph{$\bar\nu$-chaotic} if, 
	for any $r \in \nats$ and any bounded continuous real-valued functions $g_{1}, g_{2}, \ldots, g_{r}$,
	\[
	\begin{aligned}
		\lim _{N \rightarrow \infty}\EE_{X_{N,1:N}}\left\{\prod_{i=1}^{r}g_{i}\left(X_{N,i}\right)\right\}
		&= \prod_{i=1}^{r} \int g_{i}(x) \bar\nu(\dee x).
	\end{aligned}
	\]
\label{chaos definition}
\end{definition}
\end{frame}

\begin{frame}{Adapted Markov Chains are Chaotic}
\begin{assumption} 
	\label{assumption}
	%We have the following assumptions. 
	\begin{enumerate}
		\item \label{assumption1}%
		The proposal probability density $q_{h}(y, x)$ is continuous with respect to $(x, y, h)$.
		\item \label{assumption2}%
		The target distribution has a continuous probability density function $\posteriorDensity(\cdot)$.
		\item \label{assumption3}%
		Samples generated from the Markov transition kernel $T (x, y, h, \cdot, \cdot )$ satisfy $\EE \|X_{j}^{(t)}\|^{2} < \infty$ and $\EE \|Y_{j}^{(t)}\|^{2} < \infty$ for any $t \in \nats$.
	\end{enumerate}
\end{assumption}
\begin{theorem}
\label{propagation of chaos for x}
Under some mild assumptions, for any $t \in \nats$, there exists a probability 
distribution $\LimitingMeasureXt$ such that the sequence 
$\{\CurrentSamples\}_{N=1}^{\infty}$ is $\LimitingMeasureXt$-chaotic.
\end{theorem}

\end{frame}

\begin{frame}{Experiment: Neal-Funnel Shape Model}
Neal's example has support for $\log(\sigma) \in \mathbb{R}$ and $x \in \mathbb{R}^{d-1}$. The parameterization of this model is given as follows:
$$
    \log(\sigma) \sim \mathcal{N}(0, \sigma_{0}^{2}), \quad x_{i} \sim \mathcal{N}(0,\sigma).
$$
For illustrative purpose, let $\beta_{1}=\log(\sigma)$ and $\beta_{2}=x_{1}$. In our experiment, $\sigma_{0}=1$.
\begin{figure}
    \centering 
    \includegraphics[width = 0.25\textwidth]{neal_funnel.jpg}
\end{figure}
\end{frame}

\begin{frame}{Experiment: Neal-Funnel Shape Model}
\begin{figure}[t]
    \includegraphics[width=\linewidth]{neal_high_new.pdf}
    \caption{Diagnostics for Neal-funnel shape model, where 
        %$B_{\text{mean}}$ and $B_{\text{variance}}$ are computed based on Barker proposal. 
        TADDAA uses the Barker proposal. 
        Here $\mu_{i}$ and $\sigma_{i}^{2}$ denote, respectively, the mean and variance of $X_{i}$. }
    \label{fig:High Dimensional Neal Funnel Diagnostics}
\end{figure}
\end{frame}

\begin{frame}{Experiment: Neal-Funnel Shape Model}
Ablation study on $d=30$: the lower bounds become nearly constant at our proposed number of iterations $T$.
\begin{figure}
      \centering
      \includegraphics[width=\textwidth]{neal_high_iteration.pdf}
      %\caption{Another figure caption.}
\end{figure}
\end{frame}

\begin{frame}{Experiment: Logistic Regression Using Horseshoe Prior}
We use a logistic regression model with a sparsity-inducing horseshoe prior on 
\[
\begin{aligned}
    y &\mid \beta \sim \distBern(\operatorname{logit}^{-1}(X \beta)),\\
    \beta_{j} &\mid \tau, \lambda, c \sim \distNorm(0, \tau^{2} \tilde{\lambda}_{j}^{2}), \\
    \lambda_{j} &\sim \mathrm{C}^{+}(0,1), \qquad \tau \sim \mathrm{C}^{+}\left(0, \tau_{0}\right), \\
    c^{2} &\sim \distInvGam(2,8),
\end{aligned}
\]
where $y$ denotes the binary outcomes, $\tau >0 $ and $\lambda >0$ are global and local shrinkage parameters. 
\begin{itemize}
    \item $X \in \reals^{71 \times 100}$ is the features matrix.
    \item Parameter dimensionality is $d = 203$.
\end{itemize}
\end{frame}

\begin{frame}{Experiment: Logistic Regression Using Horseshoe Prior}
\begin{itemize}
    \item Mean and variance confidence intervals: capture both accurate and inaccurate marginal estimates, provide quite precise lower bounds.
    \item Computational efficiency: use $28\%$ as many gradient evaluations as VI.
\end{itemize}
     \begin{figure}
      \centering
      \includegraphics[width=0.8\textwidth, height= 6cm]{cancer_diagnostic.pdf}
      %\caption{Another figure caption.}
    \end{figure}
\end{frame}

\begin{frame}{Experiment: Logistic Regression Using Horseshoe Prior}
Reliability check: Barker and MALA pass reliability check, RWMH and HMC chains fail to mix. 
\begin{figure}
  \centering
  \includegraphics[width=0.8\textwidth]{cancer_reliability.pdf}
  %\caption{Another figure caption.}
\end{figure} 
\end{frame}


% Section 2
\section{Stationary Analysis of Fixed Learning Rate Stochastic Gradient Algorithms}
\begin{frame}{Stochastic optimization}
Consider data $\{x_{n}\}_{n=1}^{N}$ with $x_{n} \in \obsSpace$. 
For a parameter $\theta \in \reals^{D}$, observation-level differentiable loss 
$\ell : \obsSpace \times \reals^{D} \to \reals$, and regularizer $\reg : \reals^{D} \to \reals$, 
%bounded from below\fTBD{previously assumed non-negativity}, 
we aim to minimize the loss function
\begin{equation*}
\textstyle \loss(\theta) \defas N^{-1} \sum_{n=1}^{N} \ell(x_{n}, \theta) + N^{-1}\reg(\theta). 
\label{eq:loss function}
\end{equation*}
\begin{itemize}
    \item Gradient Descnet (GD): $$\textstyle\theta_{t} = \theta_{t-1} - \Lambda \nabla \textstyle \loss(\theta_{t-1})$$
    \item Stochastic Gradient Descnet (SGD): 
\begin{equation*}
\textstyle\theta_{t} = \theta_{t-1} - \Lambda G_{t}(\theta_{t-1}),
% -\Lambda B^{-1} \sum_{i \in S_{t}} \nabla \ell \left(x_{i}, y_{i}, \theta_{t-1} \right),
\label{eq: sgd update}
\end{equation*}
where $G_{t}(\theta) \defas B^{-1} \sum_{n \in S_{t}} \nabla \ell(x_{n}, \theta) + N^{-1} \nabla \reg(\theta)$ is the stochastic gradient.
    % with $\Lambda \in \reals^{D\times D}$ being a positive definite learning rate matrix.
\end{itemize}
\end{frame}

\begin{frame}{Subsampling Markov chain Monte Carlo (SGLD)}
SGLD is a Markov chain Monte Carlo (MCMC) algorithm equivalent to modifying SGD to include an additional Gaussian noise term
\begin{equation*}
\theta_{t} = \theta_{t-1} - \Lambda\,G_{t}(\theta_{t-1}) + \sqrt{2\beta^{-1} \Lambda}\,\xi_{t-1},
\label{eq: general SGLD update rule}
\end{equation*}
\begin{itemize}
    \item $\beta \in (0, \infty]$ is the inverse temperature (canonically set to $\beta = N$).
    \item $\xi_{t-1} \distiid \Norm(0, I)$.
\end{itemize}
\end{frame}

\begin{frame}{Goal}
We would like to accurately estimate the stationary covariance structure 
\begin{equation*}
\Sigma_{\theta} \defas \lim_{t \to \infty} \text{Cov} (\theta_{t}).
\end{equation*}

Characterizing the stationary covariance can help quantify 
\begin{itemize}
     \item Test loss.
     \item Escaping efficiency from a sharp minimal. 
\end{itemize}
\end{frame}

\begin{frame}{Related Work: Quadratic Loss}
	Current works assume that the loss is well-approximated by a quadratic function:
	%\begin{condition}
	%	For all $t = 1,\dots,T$, it must hold that  
	\begin{equation*}
		\textstyle\loss(\theta_{t}) \approx \tilde{\loss}(\theta_{t}) := \frac{1}{2} \rbra[\big]{\theta_{t}-\MLEN}^{\top} \hess \rbra[\big]{\theta_{t}-\MLEN} + \mathrm{const},
		%\label{eq: approximate loss function}
	\end{equation*}
	where $\hess \defas \grad^{2} \loss(\MLE)$ is the Hessian of the loss (evaluated at $\MLE$). 
\end{frame}

\begin{frame}{Related Work: Continuous-time Proxies}
The most popular proxy approach is to replace discrete dynamics of the iterative algorithm by a continuous-time the Ornstein--Uhlenbeck (OU) process
$$
	\dee \vartheta_{t} = -\Lambda \hess \vartheta_{t} \d t + \Lambda \widehat{C}^{1/2} \dee W_{t},
$$
where $W_{t}$ be a $d$-dimensional Brownian motion and $\widehat{C} = \cov(G_{1}(\MLE))$ denotes the gradient noise covariance at the minimizer.
\begin{itemize}
	\item The covariance matrix of the stationary distribution $\Sigma_{\vartheta} \defas \cov(\pi_{\vartheta})$ 
	satisfies 
	$$\Sigma_{\vartheta} \hess + \hess \Sigma_{\vartheta} = \Lambda \widehat{C}.$$
	\item \textcolor{red}{Limitation:} continuous-time proxies provide close approximation to SGD only for \textbf{small learning rates}.
\end{itemize}

\end{frame}

\begin{frame}{Related Work: Discrete-time proxies}
Assuming the loss is well-approximated by quadratic loss, the discrete-time proxy algorithm updates 
\begin{equation*}
	\label{eq:discrete-time-model-general}
	\psi_{t} = \psi_{t-1} - \frac{\Lambda}{B}\sum_{n\in S_{t}} \hess_{n} (\psi_{t-1} - \MLE),
\end{equation*}
where $\hess_{n} \defas  \grad^{2} \ell(x_{n}, \MLE)$.
\begin{itemize}
	\item \emph{Implicit} characterization of $\Sigma_{\psi}$:
	\begin{equation*}
		\Lambda \hess \Sigma_{\psi} + \Sigma_{\psi}  \hess \Lambda = \Lambda \left(\overline{C}_{\psi} + \hess \Sigma_{\psi} \hess \right)\Lambda,
		\label{eq: SGD stationary covariance}
	\end{equation*}
where $\Sigma_{\psi} \defas \cov(\pi_{\psi})$, and $\overline{C}_{\psi} \defas \E[\cov\{G_{1}(\psi_{\infty})\}]$ denotes the expected covariance of the gradient noise. 
	\item For well-specified linear model and assume $X \sim \Norm(0, A)$: 
	\begin{equation*}
		\overline{C}_{\psi} %= \lim_{t \to \infty} \mathbb{E} \left[ C(\theta_{t})\right] 
		\approx B^{-1} \left( A \Sigma_{\psi} A + \Tr \left[ A \Sigma_{\psi} \right]A + \sigma^{2} A \right).
		\label{eq: sgd noise in well-specified linear model}
	\end{equation*}
	
\end{itemize}
	
\end{frame}

\begin{frame}{Limitations of discrete-time proxies}
	\begin{itemize}
		\item Assumptions often do not hold in practice:
		\begin{itemize}
			\item Sample size $N >> D$.
			\item Mean Squared Error (MSE) loss.
			\item The model is well-specified. 
		\end{itemize}
		\item There is no guarantee that the proxy process $(\psi_{t})_{t \ge 0}$ is close 
		to the original process $(\theta_{t})_{t \ge 0}$.
	\end{itemize}
\end{frame}

\begin{frame}{A New Proxy Algorithm for Analyzing SG(L)D}
Our approach is to apply a second-order Taylor approximation to each loss term $\ell_{n}(\theta) \defas \ell(x_{n}, \theta)$:
\begin{equation*}
	\tilde\ell_{n}\left(\theta\right)  \defas \ell_{n}(\MLE) + \nabla \ell_{n}^{\top}( \MLE) (\theta-\MLE)+ (\theta-\MLE)^{\top}\grad^{2} \ell_{n}(\MLE) (\theta-\MLE),
	\label{eq: proxy algorithm framework for general loss}
\end{equation*}\pause

\begin{itemize}
	\item Minimizer $\MLE$ satisfies 
	$$
	\nabla\textstyle\loss(\MLE) = \frac{1}{N} N^{-1} \sum_{n =1}^{N} \nabla \ell(x_{n}, \theta) + N^{-1} \nabla \reg(\theta) = 0
	$$
	\item In general, 
	$$
	B^{-1} \sum_{n \in S_{t}} \nabla \ell(x_{n}, \MLE) + N^{-1} \nabla \reg(\MLE) \neq 0.
	$$
\end{itemize}
\end{frame}

\begin{frame}{Stationary Fluctuation}
Our new proxy algorithm update as follows:
\begin{equation}
	\begin{aligned}
		\psi_{t} &= \psi_{t-1} -\frac{\Lambda}{B} \sum_{n \in S_{t}} \left\lbrace \nabla \ell_{n}\big(\MLE\big) +\mathcal{J}_{n} \big(\psi_{t-1}-\MLE\big)\right\rbrace \\
		&\quad - \frac{\Lambda}{N} \nabla \reg(\psi_{t-1}) + \sqrt{2\beta^{-1} \Lambda}\,\xi_{t-1}.
	\end{aligned}
	\label{eq: proxy discrete-time update for general loss}
\end{equation}
	\begin{proposition} \label{thm:SGLD-stationary-covariance} 
		%\label{corollary: stationary covariance of SGLD}
		%Suppose the loss is well approximated by  a quadratic function \cref{eq: approximate loss function}, let $\theta_{t}$ be updated according to \cref{eq: general SGLD update rule}. 
		%The stationary noise is, for a fixed learning rate matrix $\Lambda$, the stationary covariance $\Sigma$ satisfies
		Assuming the iterates $(\psi_{t})_{t \ge 0}$ have a well-defined stationary distribution, the stationary covariance $\Sigma_{\psi}$ satisfies 
		\begin{equation*}
			\Lambda \hess \Sigma_{\psi} + \Sigma_{\psi}  \hess \Lambda = \Lambda \big(\overline{C}_{\psi} + \hess \Sigma_{\psi} \hess \big)\Lambda + 2\beta^{-1}\Lambda.
			\label{eq: SGLD stationary covariance}
		\end{equation*}
%		In particular, if $\Sigma_{\psi}$ commutes with $\overline{C}_{\psi}$ and $\Lambda$, then
%		\begin{equation} 
%			\Sigma_{\psi} = \left(\Lambda\overline{C}_{\psi} \Lambda + 2\beta^{-1}\Lambda\right)\big(\Lambda \hess + \hess \Lambda - \Lambda \hess^{2} \Lambda\big)^{-1}.
%			\label{eq: stationary covariance of SGLD}
%		\end{equation}
%		Or, to obtain covariance $\Sigma_{\psi} = \Sigma$, if $\Lambda$ commutes with $\Sigma \hess$, then set 
%		\begin{equation}
%			\begin{aligned}
%				\Lambda &= \rbra{\hess\Sigma + \Sigma \hess - 2\beta^{-1}I_{D}} \big(\overline{C}_{\psi} +  \hess \Sigma\hess\big)^{-1}.
%			\end{aligned}
%			\label{eq: discrete quadratic learning rate}
%		\end{equation}
	\end{proposition}
	
\end{frame}

\begin{frame}{Stationary Gradient Noise}
	\begin{theorem} \label{theorem: general case covariance matrix of noise}
		For the proxy algorithm, if $\reg(\theta) = \frac{1}{2} \theta^{\top}\Gamma \theta^{\top}$ and the mini-batches
		are sampled with replacement, then 
		%the expected stochastic gradient covariance at stationarity satisfies\fTBD{I changed $\mathcal{J}$ to $\hess$}
		\begin{equation*}
			\label{eq: covariance of stationary noise}
			\begin{aligned}
				\overline{C}_{\psi} =\frac{1}{B} \left( \mathcal{I} - \frac{1}{N^{2}} \Gamma \MLE \MLE^{\top} \Gamma^{\top} + \frac{1}{N} \sum_{n=1}^{N} \mathcal{J}_{n} \Sigma_{\psi} \mathcal{J}_{n} -  \mathcal{J} \Sigma_{\psi}\mathcal{J} \right),
			\end{aligned} 
		\end{equation*}
		where $\mathcal{I} \defas  \frac{1}{N} \sum_{n=1}^{N} \nabla\ell_{n}\bigl(\MLE\bigr) \nabla\ell_{n}\bigl(\MLE\bigr)^{\top}$.
		If the mini-batches are sampled without replacement, the same result holds but with the right-hand side multiplied by $(N-B)/(N-1)$. 
	\end{theorem}
\end{frame}

\begin{frame}{Wasserstein Distance}
How to assess the accuracy of our proxy algorithm? \textbf{Wasserstein Distance}
\begin{itemize}
\item \[
W_{2}(\pi, \tilde\pi) = \inf \E(\norm{\theta - \tilde\theta}^{2})^{1/2},
\]
where the infimum is over all joint distributions of $(\theta, \tilde\theta)$ such that $\theta \dist \pi$ and $\tilde\theta \dist \tilde\pi$. 
\item $W_{2}(\pi_{\theta}, \pi_{\psi}) \leq \veps$ implies that
\begin{equation*}
	\begin{aligned}
		&\label{eq:general-stdev-and-cov-error-bounds}
		|\sigma_{\theta,d} - \sigma_{\psi,d}| \leq \veps~(d = 1,\dots,D) \\
		&\norm{\Sigma_{\theta} - \Sigma_{\psi}} \le 2 \veps (\norm{\Sigma_{\theta}}^{1/2} \wedge \norm{\Sigma_{\psi}}^{1/2} + \veps).
	\end{aligned}
\end{equation*}
\end{itemize}
\end{frame}

\begin{frame}{Error Analysis}
\begin{theorem} \label{thm:wasserstein-error-bound}
	Under standard assumptions and $\Lambda = \lambda I_{D}$ for some $\lambda \in (0, 1/(2L))$,
	then, letting $\beta \defas 1 - \lambda\mu \rbra*{1 - 2\lambda L}$, $\overline{M} \defas \{N^{-1} \sum_{n=1}^{N} M_{n}^{2}\}^{1/2}$,
	and $C_{s} \defas  \E\rbra{\norm{\psi_{s} - \MLE}^{4}}$, for all $t = 1,2,\dots$,
	\begin{equation*}
	\label{eq:wasserstein-error-bound} 
	\begin{aligned}
\lefteqn{W_{2}^{2}(\pi_{\theta,t}, \pi_{\psi,t})} \\
&\le \beta^{t}W_{2}^{2}(\theta_{0}, \psi_{0})  
+ \lambda  \overline{M} \left\lbrace \frac{\lambda  \overline{M}}{2} + \frac{2}{\mu}\right\rbrace  \sum_{s=1}^{t}\beta^{t-s}C_{s-1}. \nonumber
	\end{aligned}
	\end{equation*}
\end{theorem}
\begin{corollary} \label{cor:stationary-wasserstein-error-bound} 
	Under the same assumptions stated above and with $\beta = \infty$ (i.e., for the case of SGD), if $\lambda < L/4$, then 
	there exists an explicit constant $A$ such that 
	\[
	W_{2}(\pi_{\theta}, \pi_{\psi}) \le A\frac{\lambda}{B}.
	\]
\end{corollary}
\end{frame}

\begin{frame}{Experiments: Linear Regression}
To validate our theory, we compare the predicted stationary covariance structure under the fixed learning rate obtained from other theory with others.
\begin{itemize}
	\item \emph{Simulated misspecified data.}:
	\begin{equation*}
		\label{eq: linear regression misspecified model}
		y_{n} \dist \Norm(x_{n}^{\top}\theta_{\star}, 1+\|x_{i}\|_{2}^{2}),
	\end{equation*}
	where $\theta_{\star} \dist \Norm(0, I_{D})$ is fixed and $x_{n} \distiid \Norm(0, I_{D})$.
	\item Real-world dataset: \emph{Boston housing data.}
\end{itemize}
\end{frame}

\begin{frame}{Experiments: Linear Regression}
\begin{figure}
	\centering
	\begin{minipage}[b]{0.4\textwidth}
		\centering
		\includegraphics[width=\textwidth]{linear_simulation_single.pdf} 
		%	\caption{First PDF}
	\end{minipage}
	\hspace{0.0001\textwidth}  % Space between the two PDFs
	\begin{minipage}[b]{0.4\textwidth}
		\centering
		\includegraphics[width=\textwidth]{linear_real_single.pdf} 
		%	\caption{First PDF}
	\end{minipage}
	%\caption{Compare learning rate tuning guidance suggested by different theories on a Boston Housing dataset.}
	\caption{Comparison of estimated stationary covariance structure for linear regression at $3\sigma$ confidence region on \textbf{(left)} simulated misspecified data with heteroskedastic noise 
		and \textbf{(right)} the classic Boston housing dataset with $\lambda = 0.1$ and $B = 32$.
		Our theory provides more accurate stationary covariance predictions in both cases.
	}\vspace{-1em}
	\label{Fig: linear regression fixed_lr}
	%	\label{Fig: Linear regression on synthetic dataset_fixed_lr}
\end{figure}
\end{frame}

\begin{frame}{Experiments: Poisson Regression}
\begin{itemize}
	\item \emph{Simulated misspecified data.}:
\begin{equation*}
	\label{eq: poission regression model}
	y_{n} \sim \text{Poisson}(\exp\{x_{n}^{\top}\theta_{\star}\}),
\end{equation*}
	where $\theta_{\star} \dist \Norm(0, I_{D})$ is fixed and $x_{n} \distiid \Norm(0, I_{D})$.
	\item Real-world dataset: \emph{German credit data.}
\end{itemize}
\end{frame}

\begin{frame}{Experiments: Poisson Regression}
\begin{figure}[t]
	\centering
	\begin{minipage}[b]{0.4\textwidth}
		\centering
		\includegraphics[width=\textwidth]{poisson_simulation_single.pdf} 
		%	\caption{First PDF}
	\end{minipage}
	\hspace{0.0001\textwidth}  % Space between the two PDFs
	\begin{minipage}[b]{0.4\textwidth}
		\centering
		\includegraphics[width=\textwidth]{poisson_real_single.pdf} 
		%	\caption{First PDF}
	\end{minipage}
	
	\caption{Comparison of estimated stationary covariance structure for Poisson regression at $3\sigma$ confidence region with \textbf{(left)} simulated well-specified data and \textbf{(right)} the German credit data by setting batch size $\lambda = 0.1$, and $B = 32$.
		%Our theory provides more accurate stationary covariance predictions in both cases.
	} % synthetic and real-world dataset.}
\label{Fig: Poisson regression fixed_lr}
\end{figure}
\end{frame}




% Conclusion and Further work
\section{Conclusion}
\begin{frame}
  \frametitle{Conclusion}
  \begin{itemize}
  \item We propose a diagnostic tool for VI:
\begin{itemize}
	\item supports marginal checks and is applicable to high-dimensional parameter spaces
	\item provides lower bounds on the error of specific posterior summaries
	\item is computationally efficient
	\item can be validated using a simple correlation-based reliability check
\end{itemize}
\item We propose tools for stationary covariance analysis of stochastic gradient algorithms:
\begin{itemize}
	\item has minimal checkable assumptions. 
	\item Nonasymptotic error analysis. 
\end{itemize}
  \end{itemize}
\end{frame}


% A thank you slide
\begin{frame}{\Large Acknowledgments}
	\includegraphics[width=\textwidth]{acknow.pdf} 
\end{frame}

\end{document}
